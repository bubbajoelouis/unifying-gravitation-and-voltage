\documentclass{article}
\usepackage{amsmath, amssymb, geometry}
\geometry{a4paper, margin=1in}

\title{\textbf{Unifying Gravitation and Voltage: The Recursive Radiation Source Hierarchy}}
\author{Michael Vera}
\date{\today}

\begin{document}

\maketitle

\begin{abstract}
This paper establishes a fundamental unification between Gravitation and Voltage, demonstrating that Electric Potential is stored Radiation within a recursive hierarchy of Radiation Sources. Traditional physics treats Gravitation and Electromagnetism as distinct domains, yet through the Unified Theory of Energy (UTE), we show that Voltage, defined as Electric Potential, is equivalent to stored Gravitation. We further explore how electrical circuits, like gravitational systems, are structured as nested Radiation Coordinate Systems. This work corrects the ontic vagueness caused by enforcing $n=2$ dimensional mathematics and proposes a more consistent framework for energy redistribution in electrical and gravitational fields.
\end{abstract}

\section{Introduction}
Physics has long treated Gravitation and Electromagnetism as fundamentally distinct forces. However, the Unified Theory of Energy (UTE) posits that Voltage ($V$), Electric Potential, and Gravitation ($G$) are manifestations of the same fundamental principle: stored Radiation ($R$). This work formally unifies these concepts and presents a framework where electrical circuits and gravitational systems are structured through nested Radiation Coordinate Systems.

\section{Fundamental Equations}

\subsection{Voltage as Gravitation and Stored Radiation}
We define Voltage in terms of Gravitation and stored Radiation as:
\begin{equation}
    V = G = R
\end{equation}
where:
\begin{itemize}
    \item $V$ is the Electric Potential (Voltage),
    \item $G$ is Gravitation, representing stored Radiation within a Mass Structure,
    \item $R$ is Radiation extended from a Mass Structure.
\end{itemize}

\subsection{Hierarchy of Radiation Sources}
Energy exchanges occur within a hierarchy of nested Radiation Sources:
\begin{equation}
    R_{n+1} \supset R_n
\end{equation}
where each Radiation Source contains the next:
\begin{equation}
    R_{\text{Sun}} \supset R_{\text{Earth}} \supset R_{\text{Generator}} \supset R_{\text{Circuit}}
\end{equation}
In this model, an electrical generator does not "create" energy but serves as an Electron Pump, redistributing stored Radiation within the system.

\subsection{Electron Motion as Surface Interactions}
Charge motion in a conductor is governed by:
\begin{equation}
    I = \frac{dR}{dt}
\end{equation}
where:
\begin{itemize}
    \item $I$ is current (Inertia in the UTE framework),
    \item $R$ is Radiation,
    \item The rate of change of Radiation over time describes the oscillatory nature of electron motion as a Surface Interaction.
\end{itemize}

\subsection{Conservation of Energy in Radiation Hierarchy}
Energy within this system is distributed among Gravitation, Radiation, and Inertial charge motion:
\begin{equation}
    E_{\text{total}} = G + R + I
\end{equation}
where:
\begin{itemize}
    \item $E_{\text{total}}$ is the total system energy,
    \item $G$ is stored Gravitation (Potential Energy),
    \item $R$ is extended Radiation (e.g., emitted electromagnetic energy),
    \item $I$ is Inertia, representing charge motion within the system.
\end{itemize}

\section{Implications and Corrections to Traditional Physics}
This unified framework provides corrections to traditional physics assumptions:
\begin{itemize}
    \item \textbf{Electrical Circuits as Nested Radiation Systems:} Energy losses in transmission should be understood as mismatches between hierarchical Radiation Sources rather than simple resistive losses.
    \item \textbf{Electron Motion as Recursive Surface Interactions:} Electrons do not "flow" in bulk; they oscillate within structured Radiation Sources.
    \item \textbf{Energy Generation as Redistribution:} Electrical generation is best understood as energy redistribution, governed by the same recursive laws as gravitation.
\end{itemize}

\section{Conclusion}
This work presents a unified understanding of Gravitation and Voltage within the framework of the Unified Theory of Energy. By redefining Voltage as stored Radiation and Gravitation, we remove the artificial division between gravitational and electromagnetic fields. Future work may explore how this understanding can improve energy efficiency in electrical grids and advance new methods of controlled energy transfer.

\end{document}
